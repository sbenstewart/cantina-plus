% Appendix A
\chapter{RECOMMENDATION SYSTEM}
\section{INPUT HANDLER}
In our project the module Input handler includes creation of various web services.We create a shopping portal with webservices like login,sign up,add to cart,cart manager,payment.These servuices are taken from various web providers and we collect a lot of feedback from the service consumers.
\subsection{Testcases}
\subsubsection{Input}
    Variety of methods for payment is established.Each with varying loads to determine the best.
\subsubsection{Necessity}
     This could end with the collection of feedback from all the service consumers.The collected values include reliabilty,satisafaction,throughput,quality.
\newpage
\section{RELIABILITY EVALUATOR}
•	As not every users are reliable ones, we eliminate the QOS values of untrusted users and then provide with best results.
•	A user is considered to be trusted based on the frequency of usage, the contract duration, and the threshold on the number of users. 
•	Also, the users who give ratings as all are best are also eliminated.
\subsection{Testcase 1}
\subsubsection{Input}
When there are no logs at all
\subsubsection{Output}
He or She is a reputed user.
\subsection{Testcase 2}
\subsubsection{Input}
When there are no logs in a particular service.
\subsubsection{Output}
He or She is a reputed user.
\subsection{Testcase 3}
\subsubsection{Input}
When there are two logs in a particular service.
\subsubsection{Output}
Previous values are multiplied with the reliability.
\subsection{Testcase 4}
\subsubsection{Input}
When there are more than two logs in a particular service.
\subsubsection{Output}
Algorithm is applied

\section{MISSING QoS VALUE PREDICTOR}
Description:
•	Not every users could provide the correct feedbacks and not every user is reliable.
•	This section takes care of Missing QOS values based on the Historical values provided and then predicts what  the user could have provided if he had used the system.
•	This section will improve accuracy of recommendation of services.

\subsection{Testcase 1}
\subsubsection{Input}
When there are no logs at all
\subsubsection{Output}
Random values or One will be assigned based on frequency.
\subsection{Testcase 2}
\subsubsection{Input}
When there are no logs in a particular service.
\subsubsection{Output}
Random values or One will be assigned based on frequency.
\subsection{Testcase 3}
\subsubsection{Input}
When there are two logs in a particular service.
\subsubsection{Output}
Previous values are considered for prediction.
\subsection{Testcase 4}
\subsubsection{Input}
When there are more than two logs in a particular service.
\subsubsection{Output}
Algorithm is applied to compute the output.




