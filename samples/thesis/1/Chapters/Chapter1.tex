% Chapter 1

\chapter{INTRODUCTION} % Write in your own chapter title All Chapter headings in CAPS

\section{PROJECT OVERVIEW}
In today's world several image manipulation software's are available. Manipulation of digital images has become a serious problem nowadays. There are many areas like medical imaging, digital forensics, journalism, scientific publications, etc, where image forgery can be done very easily. To find the marks of tampering in a digital image is a challenging task. Powerful image editing tools are very common and easy to use these days. This situation may cause some forgeries by adding or removing some information on the digital images. To determine whether a digital image is original or doctored is a big challenge. The detection methods can be very useful in image forensics which can be used as a proof for the authenticity of a digital image.
\section{PROJECT OBJECTIVE}
The main objective of our project is to develop a deep neural network that distinguish the original image from the tampered or forged image. An efficient algorithm to classify two types of passive image forgery detection namely Splicing and copy move forgery is used. To detect the splicing type of forgery we use  RMSprop optimiser so that neural network is trained with a very optimal learning rate. So that it will be able to converge faster and obtain an optimal result. For the copy move forgery detection a robust algorithm, that can detect forgeries in lossy image formats- such as JPEG is used.
\section{SCOPE}
To identify image threats such as copy move tampering which is most generally used by attackers in which parts of the original image is pasted on to the same image nearly matching areas, and splicing forgery were parts of two different images are merged into one,it is required to classify whether the input image is tampered or not.Not all the algorithms to detect such tampering is has proved to be very accurate and less complex. Efficient algorithms to detect these forgeries were developed which included processing the images prior to its classification so that it resulted in increased accuracy.
\section{CONTRIBUTION}
The base paper we referred used a neural network to detect the tampering in the input images given to it where features were extracted by the network and then used to classify if the image is forged or not.Our contribution is to detect if there is tampering in the images and predict its type(spliced or copy move). In order to result in more efficiency algorithm such as error level analysis is used to process the images before it is fed into the neural network so that the network extracts features more accurately.
\section{ORGANISATION OF THESIS}
Chapter 2 discusses the existing techniques for forgery detection such as using discrete cosine transform, wavelet transformations, jpeg compression artifacts etc used to identify the tampering in images,Chapter 3 gives the requirements analysis of the system. It explains the functional and non-functional requirements, constraints and assumptions made in the implementation of the system. Chapter 4 explains the overall system architecture and the design of various modules along with their complexity. Chapter 5 gives the implementation details of each module, describing the algorithms used.Chapter 6 elaborates on the results of the implemented system and gives an idea of its efficiency. It also contains information about the data set used for testing and other the observations made during testing. Chapter 7 concludes the thesis and gives an overview of its criticisms. It also states the various extensions that can be made to the system to make it function more effectively.

