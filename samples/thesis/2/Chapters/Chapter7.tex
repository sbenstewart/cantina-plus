% Chapter 7
\chapter{CONCLUSION} % Write in your own chapter title
\section{Conclusion}
In our system, Hyperledger has been used for building blockchain which is a permissioned blockchain technology. The patient details has been considered as a transaction and broadcast in the blockchain network. This triggers smart contract which parses the transaction and gets the insurance policy with regard to the patient details where it has been generated. This helps to easily find the policy under which the patient had been covered. find out the spread of heart diseases in India. This has been implemented using chain code. Blockchain makes it impossible to tamper these data which removes the control from the hospital’s central authority. Hence, the insurer or hospital cannot modify or change the insurance details
given.
\section{Future Work}
Currently, only the treatment details of the patient that are covered within the insurance policies were added to the blockchain. In future, may be all the treatment details of the patients can be added to prevent any tamper and  to make the patient details safe and secure.




