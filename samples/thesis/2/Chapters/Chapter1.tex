% Chapter 1

\chapter{INTRODUCTION} % Write in your own chapter title All Chapter headings in CAPS
\section{Problem Domain} % All Section headings in Title Case
\lipsum[]
Traditional health insurance system possesses high possible fraudulent or duplicate claims, possible financial loss during digital transactions, complex and high time consuming verification processes. This is the challenge the insurance industry faces and it possess transactional and contractual complexity. The aforementioned problem is an ideal application opportunity for Blockchain implementation. Blockchain can provide significant benefits to the private consortium and all of its participants including the insured customers. Blockchain can address the fundamental challenges of managing and dragging the distributed digital transactions of the problem scenario, as they are extremely secured, manageable and high-speed transactions. A lot of work goes into checking whether a health insurance claim is fraudulent or not. Using the consensus property of Blockchain, we can reduce the work that goes into checking if a claim is fraudulent or not.

\section{Scope} %All Sub Section headings in Title Case
\lipsum[]
Now-a-days insurance data are more important, since it can be
modified by higher authorities. Hence, the security of this insurance data becomes questionable. Hospitals may claim insurance for treatments that were not taken by patients. This happens because we completely rely on the centralized record management system of the hospital. So, we need to build a system which records the treatments and bills impossible to be counterfeited even by the higher authorities. Hence, it becomes impossible to tamper the insurance data of the patient from the eye of society. Also it takes lot of time to verify a claim made by insurer and to instantiate the transaction. Our main intention is to secure insurance data using blockchain and to implement auto insurance claim. If we implement this in India then the insurance fraudulent will be reduced. 
\section{Organization of Thesis} %All Sub Section headings in Title Case
Chapter 2 discusses the existing insurance mechanisms
and applications of blockchain in greater detail. Chapter 3 gives the
requirements analysis of the system. It explains the functional and nonfunctional requirements, constraints made in the implementation of the system. Chapter 4 explains the overall system architecture and the design of various modules along with their complexity. Chapter 5 gives the implementation details of each module, describing the algorithms used. Chapter 6 elaborates on the results of the implemented system and gives an idea of its efficiency. Chapter 7 concludes the thesis and gives an overview of its criticisms. It also states the various extensions that can be made to the system to make it function more effectively.
