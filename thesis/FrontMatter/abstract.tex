
\tab Phishing is a crime where the victim is contacted by an attacker posing as a trustworthy source and lure them into providing sensitive information like credit card details and personal identification numbers. These attacks are currently being blocked by web browsers that have a list of such phishing links. It takes several days and intense computing resources to prepare the list. Having a time lag in this process means that many victims are vulnerable at that point in time to such an attack. Inorder to make the process more efficient, the functionality required will be ported to the client side of the web browser. This makes sure that the time delay is averted and the phishing attack can be thwarted with fewer computational resources.

\tab To solve the above mentioned problem, a web browser add-on that works as a background script on the client side is to be implemented. All the background scripts required will be made cross platform compatible to make the development easier and more efficient.

\tab The main objective of the system developed is to find if the web page that the user visits is a phish or not. The work is explained as follows.

\tab The page redirect logs alongs with the page details are used to find if the site is a phish or not. And then if the site happens to be a phish, the similar looking site which this web page tries to impersonate is given. To make the repeated accesses faster, a whitelist is maintained, where all the safe sites are logged.